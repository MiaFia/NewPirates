\documentclass{article}
\usepackage[utf8]{inputenc}

\title{Python PiRates Project Proposal}
\author{Artom Obenko, Johanna Annau, Khoa Tran, Sophia Klaußner }
\date{January 2021}

\begin{document}
	
\maketitle
	
\section{Research Question(s)}
	- Do (Dutch) celebrities live longer than the average life expectancy?\\
	- Does the life expectancy differ between different countries?\\
	- Are celebrity lifespans a good approximation for life expectancy?\\
	
\section{Introduction}
Many factors influence a person's life expectancy. Diet, genetics, and country of birth are factors. Most databases however average those together and create a life expectancy for country or region, depending on the birth year. %Explain how life expectancy is calculated.
So as life expectancy is an average, it is not perfect to predict people's life span. And as so many other factors play into it, different social groups might deviate from the average. Thus, we are looking into the question if celebrities have a significantly longer life span than the life expectancy would suggest. We consider celebrities from two countries, the Netherlands and South Africa, and see if there is difference in our findings. So finally we want to find, if looking at the life spans of celebrities from a country is a good approximation for life expectancy.

\section{Methods}
We used the dataset DPwikipedia to look through the lifespans of celebrities that have a Wikipedia article. From there we filtered for real people and narrowing that down to only those that had a birth and death Year. Then we filtered for those that had the key words 'Dutch' or 'South African' in their descriptions. We chose those two countries, as the data included many entries on people from these places, so we would have enough data. Further the life expectancy in both places was still fairly different to each other. After filtering the data for each country, we found the average lifespan. As our life expectancy data only reaches back until .... we also only included data points with a birth year after that of the first data point. We then averaged all the lifespans and the life expectancies from all years since then. We assumed men and women to be 50\% of the population.
 

\section{Proposal for rest of the week}
We now have the filtered data and added it for the Netherlands into R. We also found a good dataset about life expectancies in the Netherlands and are working on incorporating that into R and filtering it for the relevant points. For the rest of the week we want to find the averages for Dutch people and then continue to do the same with those from South Africa. We still need to find data the life expectancies there and run the code to filter for it. Then we need to compare the different averages to each other, maybe graph it and determine if there are significant differences. 

\section{Limitations}
- we don't really care when someone is born and the corresponding life expectancy, so it is a very rough average\\
- we also include people from the past decades, but if they are already dead, they might not have reached the life expectancy\\
- we filtered by the word 'dutch' so it might include people that did not live there all their life etc. but across most it should be okay\\

\end{document}